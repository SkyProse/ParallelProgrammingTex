{ %section2_3
	\subsection{Gustavson-Barsis Method}
	\par When evaluating the efficiency of parallelization of a certain program running a fixed time, the speed of execution can be expressed as follows: $\left.V(p)\right|_{t=const}\;=\;\frac {w(p)}t$, where $w(p)$ – t is the total amount of work that the program manages to complete during $ t $ when using $p$ processors. Then the Expression~\eqref{parallelAcceleration:equation} for parallel acceleration will take the form:
	\begin{equation}
		\label{GustavsonAcceleration:equation}
		\left.S(p)\right|_{t=const}\;=\;\frac{V(p)}{V(1)}\;=\;\frac{w(p)}t\;:\;\frac{w(1)}t\;=\;\frac{w(p)}{w(1)}.
	\end{equation}
	\par We write the amount of work $w (1)$ as follows:
	\begin{equation}
		\label{GustavsonWork:equation}
		w(1)\;=\;w(1)\;+\;(k\;\cdot\;w(1)\;-\;k\;\cdot\;w(1))\;=\;k\;\cdot\;w(1)\;+\;(1\;-\;k)\;\cdot\;w(1),
	\end{equation}
	where $k\in\lbrack0,1)$ – is the parallelism coefficient of the program. Then the first term can be considered the amount of work that perfectly parallelizes, and the second - the amount of work that fails to parallelize when adding processors (cores).
	\par When using $p$ processors, the amount of work done $w(p)$ will obviously become larger, while it will consist of two terms:
	\begin{itemize}
		\item number of unparalleled work $(1-k)\cdot w(1)$, which does not change compared to the Formula~\eqref{GustavsonWork:equation}.
		\item amount of parallel work, the volume of which will increase by $p$ times in comparison with the formula ~\eqref{GustavnsonWork:equation},because $p$ processors will be used instead of one.
	\end{itemize}
	\par Given the above, we obtain the following expression for $w(p)$:
	\par$w(p)\;=\;p\;\cdot\;k\;\cdot\;w(1)\;+\;(1\;-\;k)\;\cdot\;w(1),$ then given the formulas~\eqref{GustavsonAcceleration:equation} we get: $\;\frac{w(p)}{w(1)}=\;\frac{p\cdot k\cdot w(1)+(1-k)\cdot w(1)}{w(1)}$, that allows you to record:
	\begin{equation}
		\left.S(p)\right|_{t=const}\;=\;S_{GB}(p)\;=\;p\;\cdot\;k\;+\;1\;-\;k
	\end{equation}
	\par The above expression is called \textbf {Gustavson-Barsis law}, which John Gustavson and Edwin Barsis formulated in 1988.
	\par
}