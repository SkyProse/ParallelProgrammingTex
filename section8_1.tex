{ %section8_1
	\subsection{Порядок выполнения работы}
	\begin{enumerate}
		\item Take OpenMP program from Laboratory research \#5, in which all the stages of calculation are parallelized, as an initial one. Make sure that this program correctly implements simultaneous access to the common variable used to output the program completion percentage to the console.
		\item Change the source code so that the "POSIX Threads" standard is used instead of OpenMP directives:
			\begin{itemize}
				\item change only one stage (Generate, Map, Merge, Sort), which is a bottleneck, as well as the output of the program completion percentage to the console function to get \textbf{satisfactory mark};
				\item change the entire program (you are allowed to use "schedule static" as the loops schedule) to get \textbf{good} and \textbf{excellent mark};
				\item parallelize at least one cycle by manually implementing "schedule dynamic" or "schedule guided" schedule to get \textbf{excellent mark}.
			\end{itemize}
		\item Conduct experiments and compare the results of two parallel programs ("OpenMP" and "POSIX Threads"). The comparison must describe the following aspects of both programs (for different $N$):
			\begin{itemize}
				\item total time for solving the task;
				\item parallel acceleration;
				\item the percentage of time spent at each stage of the calculation ("normalized chart with areas and accumulation");
				\item the number of code lines added during parallelization, as well as a rough estimate of the time spent on parallelization (programmer's overhead);
				\item other aspects that you have found out yourself (\textbf{Mandatory point});
			\end{itemize}
	\end{enumerate}
}