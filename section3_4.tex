{ %section3_4
	\subsection{OpenCL Technology}
	\label{OpenCL:section}
	\par\textbf{Brief description of the technology.} OpenCL is a framework for writing computer programs related to parallel computing on various graphic and central processors, as well as FPGA. OpenCL includes a programming language that is based on the C99 language standard, and an application programming interface. OpenCL provides concurrency at the instruction level and data level and is an implementation of the GPGPU technique. OpenCL is a fully open standard, its use is not subject to licensing fees. Using this technology, heterogeneous parallel computing can be performed (distribute tasks between different devices).
	\par As we already know, it is possible to parallelize a program by tasks between a small number of productive cores (processors of modern PCs) or by data between thousands of simple slow cores (computing cores of modern GPUs). It is for problems solved using data parallelization that OpenCL is used.
	\par\textbf{OpenCL Technology Architecture} OpenCL divides two types of devices: \textit{host}, which controls the common logic, and \ textit{device}, which perform the calculations. The \textit{host} is usually the central processor, while the \textit{device} is the GPU and other devices. \textit{Device} divided into compute modules \textit{computer units}, which in turn consist of processing elements \textit{(processing elements)} (Figure~\ref{OpenCLArchitecture:image}). Direct calculations are performed in the processing elements of the device.
	\begin{figure}[H]
		\includegraphics[width=1\linewidth]{OpenCLArchitecture}
		\caption{\textit{OpenCL Architecture}}
		\label{OpenCLArchitecture:image}
	\end{figure}
	\par Physically, a \textit{computer unit} is a \textit{work-group} that consists of \textit{work-item} cells which perform calculations (Figure~\ref{OpenCLWorkGroup:image}).
	\begin{figure}[H]
		\includegraphics[width=1\linewidth]{OpenCLWorkGroup}
		\caption{\textit{OpenCL  Architecture – the structure of work-group element}}
		\label{OpenCLWorkGroup:image}
	\end{figure}
	\par Commands in OpenCL form a queue. \textit {Host} routes commands to devices. These teams queue up similar teams. You can implement a queue in order and without compliance.
ID functions \textit{work-group} and \textit{work-item} are shown on Figure~\ref{OpenCLWorkGroupItemExample:image}.
	\begin{figure}[H]
		\includegraphics[width=1\linewidth]{OpenCLWorkGroupItemExample}
		\caption{\textit{OpenCL. Operation with work-group and work-item}}
		\label{OpenCLWorkGroupItemExample:image}
	\end{figure}
	\par\textbf{Types of memory in OpenCL devices.} To interact with data, the programmer can use different levels of memory. On the Figure~\ref{OpenCLMemory:image}  we can see the following types of memory exist:
	\begin{itemize}
		\item Private memory. The fastest of all kinds. Exclusive for every item of work.
		\item Local memory. It can be used by the compiler with a large number of local variables in any function. In terms of speed characteristics, local memory is much slower than register memory. Access from work items in one work-group.
		\item Constant memory. Fast enough of the available GPUs. It is possible to write data from the host, but at the same time, it is only possible to read data within the entire GPU. Dynamic allocation, unlike global memory, is not supported in constant memory.
		\item Global memory. The slowest type of memory available from the GPU. Global variables can be selected using the specifier, as well as dynamically. The global memory is mainly used to store large amounts of data received on the device from the host’s. In algorithms requiring high performance, the number of operations with global memory must be minimized.
	\end{itemize}	
	The programmer must explicitly give copies between different memory.
	\begin{figure}[H]
		\includegraphics[width=1\linewidth]{OpenCLMemory}
		\caption{\textit{Types of memory in OpenCL devices}}
		\label{OpenCLMemory:image}
	\end{figure}

	
		\par An OpenCL program may include the following sequence of actions:
		\begin{enumerate}
			\item\textbf{Platform choice:} clGetPlatformIDs, clGetPlatformInfo
			\item\textbf{Device choice:} clGetDeviceIDs, clGetDeviceInfo
			\item\textbf{Creating Computational Context:} clCreateContextFromType
			\item\textbf{Create a command queue:} clCreateCommandQueueWithProperties
			\item\textbf{Buffer Allocation:} clCreateBuffer
			\item\textbf{Creating a program object:} clCreateProgramWithSource
			\item\textbf{Code compilation:} clBuildProgram
			\item\textbf{Kernel Creation:} clCreateKernel
			\item\textbf{Working with Work-Group:} clGetKernelWorkGroupInfo 
			\item\textbf{Kernel execution:} clEnqueueNDRangeKernel 
			\item\textbf{Kernel Pending:} clWaitForEvents 
			\item\textbf{Profiling:} clGetEventProfilingInfo
		\end{enumerate}
Next, we consider some of these actions in more detail.
	\par\textbf{Platform, device, and context selection.}  \textit{(Context)} is required to manage OpenCL objects and resources. All OpenCL resources are context bound. The following data is associated with the context (Figure~\ref{OpenCLContext:image}):
		\begin{itemize}
		\item devices
		\item object objects
		\item kernel
		\item memory objects
		\item command queues.
		\end{itemize}
	\begin{figure}[H]
		\includegraphics[width=1\linewidth]{OpenCLContext}
		\caption{\textit{OpenCL Architecture – Context}}
		\label{OpenCLContext:image}
	\end{figure}
It is possible to obtain information about the platform and the computing core using special functions, then to create a context:
		\begin{itemize}
			\item\textit{clGetPlatformInfo()} – contains information about the platform on which the program runs
			\item\textit{clGetDeviceDs()} – contains information about connected devices
			\item\textit{clGetDeviceInfo()} – contains information about this device: its type, compatibility, etc.
		\end{itemize}
Context can be created using the function \textit{clCreateContext()}. Here is an example of its creation:
	\begin{figure}[H]
		\includegraphics[width=1\linewidth]{OpenCLContextExample}
		\caption{\textit{OpenCL Architecture – Creating a context}}
		\label{OpenCLContextExample:image}
	\end{figure}
In line 3 we get the platform ID, in line 7 the ID of the first GPU on this platform, in line 11 we create a context for this device. You can read more about the arguments accepted by these functions in the documentation. There is also a function \textit{clCreateContextFromType()} to create the context associated with devices of a particular type.
	\par\textbf{Kernel.} A kernel is a function that is part of a program and runs in parallel on a device. The kernel is an analog of a stream function. The part running on the device consists of a set of cores declared with a \texttt{qualifier\textbf{\textunderscore \textunderscore kernel}}. Kernel compilation can be performed at runtime using API functions. Work within the same work group is performed simultaneously by all work items.
	\par When writing the kernel, you can use the following qualifiers for variables:
		\begin{itemize}
			\item\texttt{\textunderscore \textunderscore global} or \texttt{global} – data in global memory.
			\item\texttt{\textunderscore \textunderscore constant} or \texttt{constant} – data in constant memory.
			\item\texttt{\textunderscore \textunderscore local} or \texttt{local} – data in local memory.
			\item\texttt{\textunderscore\textunderscore private }or \texttt{private} – data in private memory.
			\item\texttt{\textunderscore \textunderscore read\textunderscore only} and \texttt{\textunderscore \textunderscore write\textunderscore only} – access mode qualifiers.
		\end{itemize}
	\par We can compile kernel code using functions \textit{clCreateProgramWithSource()}, \textit{clBuildProgram()} and \textit{clCreateKernel()}. An example of compiling and running a program for multiplication of two arrays is shown in Figure \ref{OpenCLKernelExample:image}.
	\begin{figure}[H]
		\includegraphics[width=1\linewidth]{OpenCLKernelExample}
		\caption{\textit{OpenCL – compiling and running the kernel}}
		\label{OpenCLKernelExample:image}
	\end{figure}
	\sloppy You can read more about other features of OpenCL technology %%on the resource \url{http://docplayer.ru/37490743-Programmirovanie-na-opencl.html} and %%
	in official documentation.
	\begin{enumerate}
		\sloppy
		\item OpenCL – official site:\url{http://www.khronos.org/opencl/}
		\item Intel OpenCL: \url{http://software.intel.com/en-us/articles/intel-opencl-sdk/}
		\item NVIDIA OpenCL: \url{https://developer.nvidia.com/cuda-zone}
		\item AMD OpenCL: \url{http://www.amd.com/us/products/technologies/stream-technology/opencl/Pages/opencl.aspx}
	\end{enumerate}
}