{ %section5_1
	\subsection{Work sequence}
	\begin{enumerate}
		\item In the source code of the program obtained as a result of laboratory research \#1, replace all loops with calls of mathematical functions at the Map and Merge stages with vector analogs from the «AMD Framewave» library (\url{http://framewave.sourceforge.net}). Make sure that a specific Framewave function is marked as MT (Multi-Threaded), i.e. parallelized, when selecting it. The full list of available functions can be found here: \\{\small \url{http://framewave.sourceforge.net/Manual/fw_section_060.html#fw_section_060}}. For example, the Framewave function min has only SSE2 in the list of supported technologies, but not MT. 
			\par\textit{Note:} the choice of the Framewave library is not mandatory, you can use any other parallel library, if it has the necessary functions parallelized. For example, you can use ATLAS (you need to learn trottling and energy saving for this library, as well as to deal with the mechanism of changing the number of threads) or Intel Integrated Performance Primitives.
		\item Add a call to the Framewave function \texttt{SetNumThreads(M)} to the beginning of the program to set the number of threads created by the parallel library that are used when executing parallel Framewave functions. Set number M from the command-line parameter (\texttt{argv}) for ease of experiments automatization.
		\item Compile the program without using the automatic parallelization options used in laboratory research \#1. Conduct the experiments with the resulting program for the same values $N_1$ and $N_2$, which were used in laboratory research \#1, at $M=1,\;2,\;…,\;K$, where $K$ is the number of processors (cores) on the experimental stand.
		\item Compare the obtained results with the results of laboratory research~\#1: show how the program execution time, parallel acceleration, and parallel efficiency have changed on the graphs.
		\item Write a report on the work performed.
		\item Be ready to answer questions on the presentation.
		\item\textbf{Optional task \#1} (to get good and highest mark). Investigate parallel acceleration for different values of $M > K$, i.e. estimate virtualization overhead when creating a large number of threads. Give a graph of the processor (cores) loading during the execution of the program at $N=N_2$ for all used $M$. to illustrate that the program is really parallelized. You can write a script or just make a screenshot of the Task Manager, specifying the start and the end points of the experiment on the screenshot to get a graph (you need to give the text of the script or the name of the used Manager in the report).
		\item\textbf{Optional task \#2} (to get excellent mark). You should perform this task only after the previous task has been completed. Calculate the parallelization coefficient for all experiments and plot it on graphs using Amdahl's law. Explain the results.
	\end{enumerate}
}