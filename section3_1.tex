{ %section3_1
	\subsection{Debugging Parallel Programs}
	\par Parallel program debugging tools are built into most popular integrated development environments (IDEs), for example: Visual Studio, Eclipse CDT, Intel Parallel Studio, etc. These tools include convenient visualization of timing diagrams of thread execution, automatic search for suspicious program sections in which data races and deadlocks can be observed.
	\par Despite the effectiveness of existing debugging tools, there are significant difficulties when working in a debugger with a parallel program, because for its correct functioning, the debugger adds additional instructions to the machine code of the source parallel program that change the timing diagram of thread execution in relation to each other. This can lead to situations when during the testing of the program in the debugger there are no data races and deadlocks that will fully appear when the Release version of the program is launched.
	\par Also, during debugging of a multi-threaded program, it should be borne in mind that its behavior (both during regular operation and during debugging) can significantly differ when using a single-core and multi-core processor. When several threads are launched on a single-core machine, they will be executed in time-sharing mode, i.e. sequentially. This means that in this case many problems with shared access to memory and ensuring coherence of caches inherent in multi-core systems will not be observed. In addition, when debugging a program on a single-core system, a programmer can use implicit techniques to ensure the sequence of operations.
	\par For example, a programmer may incorrectly assume that when executing a high-priority thread, a low-priority thread cannot take over the processor. This assumption is correct only in a single-core system, because in the presence of several cores and a small number of high-priority threads, a situation may well be observed when a low-priority thread takes possession of one of the cores, while the high-priority thread is operating on the neighboring core.
	\par
}