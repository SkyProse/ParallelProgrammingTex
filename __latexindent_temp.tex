{ %introduction
	\phantomsection
	\section*{Introduction}
	\addcontentsline{toc}{section}{Introduction}
	\par Currently, most microprocessors are multi-core. This applies not only to desktop computers, but also to mobile phones and tablets (so far, only embedded computing systems are an exception). To fully realize the potential of a multi-core system, a programmer needs to use special methods of parallel programming, which are becoming increasingly popular in industrial programming. However, parallel programming methods are significantly more difficult to master than traditional sequential program writing methods.
	\par The purpose of this study book is to describe practical tasks (laboratory work) that can be used to consolidate the theoretical knowledge gained as part of a lecture course on parallel programming technologies. In addition, the book summarizes the basic principles of parallel programming.%, при этом теоретический материал даётся тезисно и поэтому для полноценного освоения требуется использовать конспекты лекций по соответствующей дисциплине.
	\par While programming multi-threaded applications, you have to resolve conflicts that arise when simultaneously accessing the shared memory of several threads. The following three conceptually different approaches are currently used to synchronize simultaneous access to shared memory:
	\begin{enumerate}
		\item\textbf{Explicit use of blocking primitives}\quad(mutexes, semaphores, condition variables). This approach historically appeared first and is now the most common and supported in most programming languages. The disadvantage of this method is a rather high entry threshold, since the programmer is required to manage blocking primitives in the  "manual mode"', tracking conflict situations when accessing shared memory.
		\item\textbf{Software Transactional Memory(STM)}. This method is easier to learn and use than the previous one, however it still has limited support in compilers, and it will also be able to fully manifest itself with the wider distribution of processors with hardware support for STM.
		\item\textbf{Non-blocking algorithms}\quad(lockless, lock-free,\\wait-free algorithms). This method implies a complete rejection of the use of blocking primitives with the help of complex algorithmic tricks. Moreover, for the correct functioning of the non-blocking algorithm, it is required that the processor supports special atomic (conflict-free) operations of the form "compare and exchange'' (cmpxchg, "compare and swap''). Currently, most processors have this type of operation as part of the instruction system (with rare exceptions, for example: ''SPARC 32 '').
	\end{enumerate}
	\par The methodological manual proposed is devoted to the first of the listed methods, since he received the greatest coverage in literature and the greatest application in industrial programming. Two other methods may be the subject of in-depth training courses on parallel computing.
%%	\parАвторы ставили целью предложить читателям изложение основных концепций параллельного программирования в сжатой форме в расчёте на самостоятельное изучение пособия в течение двух-трёх месяцев. При использовании пособия в технических вузах рекомендуется приведённый материал использовать в качестве односеместрового учебного курса в рамках %бакалаврской
 %%подготовки студентов по направлению подготовки ''Программная инженерия'' или смежных с ней.% Однако приводимые примеры практических заданий могут быть при желании адаптированы для использования в магистерских курсах.

}